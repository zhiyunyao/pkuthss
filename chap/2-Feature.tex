\chapter{模版功能}
本章对模版提供的功能和配置项进行介绍说明。

\section{文档类选项}
\label{sec:option}

除非特别说明,否则这一节提到的选项中都是不带“\verb|no|”的版本被启用。

\begin{itemize}
    \item \textbf{\texttt{[no]blind}}:
        是否按照盲审格式编译,使用盲审封面,不包含“发表论文信息”、“致谢”和“声明”章节。\emph{%
            pkuthss 文档类默认启用 \texttt{noblind} 选项,
            即不按照盲审格式编译%
        }。

    \item \textbf{\texttt{[no]uppermark}}:
        是否在页眉中将章节名中的小写字母转换为大写字母。
        就目前而言,这样的转换存在着一些较为严重的缺陷\footnote{%
            准确地说是 \texttt{\string\MakeUppercase} 宏的问题:
            其在某些地方的转换不够健壮,
            例如 \texttt{\string\cite\string{ctex\string}}
            会被转换成 \texttt{\string\cite\string{CTEX\string}}。%
        },因此不建议使用。
        基于上述考虑,\emph{%
            pkuthss 文档类默认启用 \texttt{nouppermark} 选项,
            即不在页眉中使用大写的章节名%
        }。

    \item \textbf{\texttt{[no]pkufont}}:
        是否根据学校对论文格式的要求\cite{pku-thesisstyle}%
        将西文字体改为类似于 Times New Roman / Arial 的字体。

    \item \textbf{\texttt{[no]pkufoot}}\footnote{%
            此选项等价于 1.6.4 及以前版本 pkuthss-extra 宏包的
            \texttt{[no]footfmt} 选项;
            更改名称是为了使文档类选项名更加规则。%
        }:
        是否根据学校对论文格式的要求\cite{pku-thesisstyle}%
        修改和脚注相关的一些格式。
        具体地说,启用 \verb|pkufoot| 选项后会进行以下几项设置:
    \begin{itemize}
        \item 脚注参用带圈的编号。
        \item 页脚中脚注编号使用正文(而非上标)字体。
        \item 页脚中脚注编号和脚注文本之间默认间隔一个空格。
    \end{itemize}

    \item \textbf{\texttt{[no]pkuspace}}:
        是否根据学校对论文格式的要求\cite{pku-thesisstyle}%
        修改排版中的一些间距及相关设置。
        具体地说,启用 \verb|pkuspace| 选项后会按
        \parencite{pku-thesisstyle} 中的要求修改以下几项设置:
    \begin{itemize}
        \item 正文的行距。
        \item 目录中条目的缩进方式。
        \item 图表标题的字号,以及标题中编号和标题文字之间的间隔方式
    \end{itemize}

    \item \textbf{\texttt{[no]spacing}}\footnote{%
            因为代码重构的缘故,
            此选项同时提供 1.5.5 及以前版本 pkuthss-extra 宏包
            中 \texttt{[no]tightlist} 选项所提供的功能。%
        }:
        是否采用一些常用的调整间距的额外版式设定。
        具体地说,启用 \verb|spacing| 选项后会进行以下几项设置:
    \begin{itemize}
        \item 调用 setspace 宏包以使某些细节处的空间安排更美观。
        \item 采用比 \hologo{LaTeX} 默认设定更加紧密的枚举环境%
            \footnote{%
                在枚举环境(itemize、enumerate 和 description)中,
                每个条目的内容较少时,条目往往显得稀疏;
                在参考文献列表中也有类似的现象。
                启用 \texttt{spacing} 选项后,
                将去掉这些环境中额外增加的(垂直)间隔。%
            }。
        \item 调整枚举环境的缩进,以适应中文排版中的习惯。
    \end{itemize}

    \item \textbf{\texttt{[no]spechap}}\footnote{%
            “spechap”是“\textbf{spec}ial \textbf{chap}ter”的缩写。%
        }:
        是否启用第 \ref{sec:misc} 小节中介绍的 \verb|\specialchap| 命令。

    \item \textbf{\texttt{[no]pdftoc}}\footnote{%
            此选项部分等价于 1.4 alpha2 及以前版本 pkuthss-extra 宏包
            的 \texttt{[no]tocbibind} 选项。
            因为 tocbibind 宏包和 biblatex 宏包冲突,%
            pkuthss 文档类不再调用 tocbibind 宏包。%
        }:
        启用 \verb|pdftoc| 选项后,
        用 \verb|\tableofcontents| 命令生成目录时会自动添加“目录”的 pdf 书签。

    \item \textbf{\texttt{[no]pdfprop}}:
        是否自动根据设定的论文文档信息(如作者、标题等)
        设置生成的 pdf 文档的相应属性。\emph{%
            注意:该选项实际上是在 \texttt{\string\maketitle} 时生效的,
            这是因为考虑到
            通常用户在调用 \texttt{\string\maketitle} 前
            已经设置好所有的文档信息。
            若用户不调用 \texttt{\string\maketitle},
            则须在设定完文档信息之后自行调用
            第 \ref{sec:misc} 小节中介绍的
            \texttt{\string\setpdfproperties} 命令以完成
            pdf 文档属性的设定。%
        }

    \item \textbf{其余文档类选项}:%
        pkuthss 文档类以 ctexbook 文档类为基础,
        其接受的其余所有文档类选项均被传递给 ctexbook。
        其中可能最常用的选项是 \verb|fontset| 和 \verb|zihao|,
        它们选择中文字体和默认字号。详见第 \ref{sec:ctex} 小节。
\end{itemize}

\section{文档信息设定}
\label{sec:info}

这一类命令的语法为
\begin{Verbatim}
  \commandname{具体信息} % commandname 为具体命令的名称。
\end{Verbatim}

这些命令总结如下:
\begin{itemize}
    \item \texttt{\bfseries\string\cthesisname}:论文类别的中文名;
    \item \texttt{\bfseries\string\thesiscover}:封面显示的论文类别;
    \item \texttt{\bfseries\string\ctitlelines}:封面论文标题的下划线行数,设置为0则不在封面显示标题;
    \item \texttt{\bfseries\string\ctitle}:设定论文中文标题,长标题用“\backslash\backslash”强制换行;
    \item \texttt{\bfseries\string\cauthor}:设定作者的中文名;
    \item \texttt{\bfseries\string\studentid}:设定作者的学号;
    \item \texttt{\bfseries\string\school}:设定作者的学院名;
    \item \texttt{\bfseries\string\cmajor}:设定作者专业(二级学科)的中文名;
    \item \texttt{\bfseries\string\direction}:设定作者的研究方向;
    \item \texttt{\bfseries\string\cmentorlines}:封面“导师”部分的行数;
    \item \texttt{\bfseries\string\cmentor}:设定导师的中文名;
    \item \texttt{\bfseries\string\degreetype}:
        设定学位类型(1 为学术学位,2 为专业学位,设为 0 则不显示学位类型);
    \item \texttt{\bfseries\string\date}:设定日期;
    \item \texttt{\bfseries\string\ckeywords}:设定中文关键词;
    \item \texttt{\bfseries\string\etitle}:设定论文西文标题;
    \item \texttt{\bfseries\string\eauthor}:设定作者的西文名;
    \item \texttt{\bfseries\string\emajor}:设定作者专业(二级学科)的西文名;
    \item \texttt{\bfseries\string\ementor}:设定导师的西文名;
    \item \texttt{\bfseries\string\ekeywords}:设定西文关键词;
    \item \texttt{\bfseries\string\discipline}:设定一级学科(双盲评审用)。
    \item \texttt{\bfseries\string\blindid}:设定论文编号(双盲评审用);
\end{itemize}

例如,如果要设定专业为“化学”(“Chemistry”),则可以使用以下命令:
\begin{Verbatim}
\cmajor{化学}
\emajor{Chemistry}
\end{Verbatim}

\section{摘要}
\label{sec:abstract}

\texttt{\bfseries cabstract} 和 \texttt{\bfseries eabstract} 环境用于编写
中英文摘要。
用户只须要写摘要的正文;标题、作者、导师、专业等部分会自动生成,盲审模式下这些信息也会自动隐藏。

如论文工作受到基金资助,需要在中文摘要第一页的页脚处标注:本研 究得到某某基金(编号:xxx)资助。


\section{目录、表格索引、插图索引}
\label{sec:directory}

目录使用 \texttt{\bfseries \string\tableofcontents} 命令生成。
表格索引使用 \texttt{\bfseries \string\listoftables} 命令生成。
插图索引使用 \texttt{\bfseries \string\listoffigures} 命令生成。

\section{主要符号对照表}
\label{sec:denotation}

参考\verb|chap/Denotation.tex|即可,在\verb|denotation|环境下,使用\verb|\item[X] Y|分别表示符号及其说明。

已知问题: 符号处不能输入中括号$[$,$]$。

\section{参考文献}
\label{sec:bibtex}

参考文献根据写作指南使用\verb|gb7714-2015|bibstyle进行管理,具体引用命令与日常使用类似,\verb|\cite{}|,\verb|\citet{}|,\verb|\citeauthor{}|,具体用法见相应文档\footnote{\url{https://github.com/hushidong/biblatex-gb7714-2015}}。

例如\verb|\cite{devlin2018bert}|=\cite{devlin2018bert},\verb|\citeauthor{gut2013probability}|=\citeauthor{gut2013probability},...
相对于的bib文件的书写基本上直接用Google Scholar拷贝的BibTex即可,部分属性按提示进行微调。
\begin{Verbatim}
    \usepackage[backend=biber,style=gb7714-2015]{biblatex}
\end{Verbatim}

\section{其他}
\label{sec:misc}

《北京大学研究生学位论文写作指南》~\cite{pku-thesisguide}提到“一般情况下,不建议使用三级及以上节标题”。
对应到本模板,即不建议使用命令 \texttt{\bfseries\string\subsubsection\{\}}。

\texttt{\bfseries\string\specialchap\{\}} 命令
用于开始不进行标号但计入目录的一章,
并合理安排其页眉。\emph{%
    注意:须要启用 \texttt{spechap} 选项才能使用此命令。
    另外,在此章内的节或小节等命令应使用带星号的版本,
    例如 \texttt{\string\section\string*\{\}} 等,
    以免造成章节编号混乱。%
}%
例如,本文档中的“北京大学学位论文原创性声明和使用授权说明”一章就是用 \verb|\specialchap{北京大学学位论文原创性声明和使用授权说明}|
这条命令开始的。%

\texttt{\bfseries\string\setpdfproperties} 命令
用于根据用户设定的文档信息自动设定生成的 pdf 文档的属性。
此命令会在用户调用 \verb|\maketitle| 命令时被自动调用,
因此通常不需要用户自己使用;
但用户有时可能不须要输出封面,从而不会调用 \verb|\maketitle| 命令,
此时就须要在设定完文档信息之后调用 \verb|\setpdfproperties|。%
\emph{注意:须要启用 \texttt{pdfprop} 选项才能使用此命令。}
