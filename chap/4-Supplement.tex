\chapter{常见需求实现}
本章介绍一些模板功能之外的常见需求的实现方法。
 
\section{表格}
\label{sec:table}

一般学术论文需要使用三线表(如表~\ref{tab:example-table-basic}),需要依赖宏包\verb|booktabs|,使用\verb|\toprule|,\verb|\midrule|,\verb|\bottomrule|控制三线。
此外表序和表名位于表格的上方。
如果需要对表格内进行脚注,可通过\texttt{minipage}中嵌套\texttt{tabular}来实现,具体可参考Stack Overflow\footnote{\url{https://stackoverflow.com/questions/2888817/footnotes-for-tables-in-latex}}。

如需要注明表格中数据来源,则可使用类似的方式,见表~\ref{tab:example-table-source-foot}。

\begin{table*}[htb]
    \centering
    \begin{minipage}[t]{0.55\linewidth} %
        \caption[表格脚注样例表]{表格脚注样例表。表名可通过中括号添加缩略名。}
        \label{tab:example-table-basic}
        \begin{small}
        \begin{tabular}{@{}lccccc@{}}
         \toprule[1.5pt]
         & \textbf{X} & \textbf{Y} & \textbf{Z} & \textbf{N} & \textbf{M} \\
         \midrule[1pt]
            默认        & 99.99 & 99.99 & 99.99 & 99.99\footnote{表格中的脚注1} & 99.99 \\
          \quad w/o X   & 99.99 & 99.99 & 99.99 & 99.99 & 99.99 \\
          \quad w/o Y   & 99.99 & 99.99 & 99.99 & 99.99 & 99.99 \\
          \quad w/o Z   & 99.99\footnote{表格中的脚注2} & 99.99 & 99.99 & 99.99 & 99.99 \\
          \quad w/o N   & 99.99 & 99.99 & 99.99 & 99.99 & 99.99 \\
          \quad w/o M   & 99.99 & 99.99 & 99.99 & 99.99 & 99.99 \\
          \bottomrule[1.5pt]
        \end{tabular}
        \end{small}
    \end{minipage}
\end{table*}

\begin{table*}[htbp]
   \centering
   \caption[数据来源注释表]{表格数据来源注释样例表。}
   \label{tab:example-table-source-foot}
   \begin{minipage}[t]{0.9\textwidth}
   \begin{small}
   \begin{tabular}{@{}l|ccc|ccc@{}}
   \toprule
   \multirow{2}{*}{\textbf{Model}} & \multicolumn{3}{c|}{\textbf{数据集A}} & \multicolumn{3}{c}{\textbf{数据集B}} \\ \cmidrule(l){2-7} 
    & \textbf{指标a}(\%) & \textbf{指标b}(\%) & \textbf{指标c} & \textbf{指标a} (\%) & \textbf{指标b}(\%) & \textbf{指标c} \\ \midrule
      \citet{devlin2018bert}      &99.99  & 99.99  & 99.99  &99.99  & 99.99  & 99.99  \\
      \citet{yang2019xlnet}      &99.99  & 99.99  & 99.99  &99.99  & 99.99  & 99.99  \\
    \bottomrule
   \end{tabular}\\[6pt]
   \footnotesize 注:数据来源XXXXXX。\\
   \end{small}
   \end{minipage}
\end{table*}

当表格较大,不能在一页内打印时,可以“续表”的形式另页打印,可使用宏包\verb|longtable|实现,如表~\ref{tab:example-table-continue}。

{\begin{longtable}[c]{c*{7}{r}}
    \caption[续表]{续表样例表。}
    \label{tab:example-table-continue}\\
    \toprule[1.5pt]
     \multicolumn{1}{c}{年龄} & 性别 & \multicolumn{1}{c}{cp} & \multicolumn{1}{c}{静息血压} & \multicolumn{1}{c}{chol}
    & \multicolumn{1}{c}{空腹血糖>} & \multicolumn{1}{c}{restecg} & \multicolumn{1}{c}{thalachh} \\
    \multicolumn{1}{c}{(岁)} & & \multicolumn{1}{c}{胸痛型}&
    \multicolumn{1}{c}{毫米汞柱}& \multicolumn{1}{c}{胆固醇}& \multicolumn{1}{c}{
       120 mg/dl}& 静息状态 & 最大心率 \\\midrule[1pt]
    \endfirsthead
    \multicolumn{8}{c}{续表~\thetable\hskip1em 续表样例表。}\\
    \toprule[1.5pt]
     \multicolumn{1}{c}{年龄} & 性别 & \multicolumn{1}{c}{cp} & \multicolumn{1}{c}{静息血压} & \multicolumn{1}{c}{chol}
    & \multicolumn{1}{c}{空腹血糖>} & \multicolumn{1}{c}{restecg} & \multicolumn{1}{c}{thalachh} \\
    \multicolumn{1}{c}{(岁)} & & \multicolumn{1}{c}{胸痛型}&
    \multicolumn{1}{c}{毫米汞柱}& \multicolumn{1}{c}{胆固醇}& \multicolumn{1}{c}{
       120 mg/dl}& 静息状态 & 最大心率 \\\midrule[1pt]
    \endhead
    \hline
    \multicolumn{8}{r}{续下页}
    \endfoot
    \endlastfoot
    63 & 1 & 3 & 145 & 233 & 1 & 0 & 150 \\
    37 & 1 & 2 & 130 & 250 & 0 & 1 & 187 \\
    41 & 0 & 1 & 130 & 204 & 0 & 0 & 172 \\
    56 & 1 & 1 & 120 & 236 & 0 & 1 & 178 \\
    57 & 0 & 0 & 120 & 354 & 0 & 1 & 163 \\
    57 & 1 & 0 & 140 & 192 & 0 & 1 & 148 \\
    56 & 0 & 1 & 140 & 294 & 0 & 0 & 153 \\
    44 & 1 & 1 & 120 & 263 & 0 & 1 & 173 \\
    52 & 1 & 2 & 172 & 199 & 1 & 1 & 162 \\
    57 & 1 & 2 & 150 & 168 & 0 & 1 & 174 \\
    54 & 1 & 0 & 140 & 239 & 0 & 1 & 160 \\
    48 & 0 & 2 & 130 & 275 & 0 & 1 & 139 \\
    49 & 1 & 1 & 130 & 266 & 0 & 1 & 171 \\
    64 & 1 & 3 & 110 & 211 & 0 & 0 & 144 \\
    63 & 1 & 3 & 145 & 233 & 1 & 0 & 150 \\
    37 & 1 & 2 & 130 & 250 & 0 & 1 & 187 \\
    41 & 0 & 1 & 130 & 204 & 0 & 0 & 172 \\
    56 & 1 & 1 & 120 & 236 & 0 & 1 & 178 \\
    57 & 0 & 0 & 120 & 354 & 0 & 1 & 163 \\
    57 & 1 & 0 & 140 & 192 & 0 & 1 & 148 \\
    56 & 0 & 1 & 140 & 294 & 0 & 0 & 153 \\
    44 & 1 & 1 & 120 & 263 & 0 & 1 & 173 \\
    52 & 1 & 2 & 172 & 199 & 1 & 1 & 162 \\
    57 & 1 & 2 & 150 & 168 & 0 & 1 & 174 \\
    54 & 1 & 0 & 140 & 239 & 0 & 1 & 160 \\
    48 & 0 & 2 & 130 & 275 & 0 & 1 & 139 \\
    49 & 1 & 1 & 130 & 266 & 0 & 1 & 171 \\
    64 & 1 & 3 & 110 & 211 & 0 & 0 & 144 \\
    49 & 1 & 1 & 130 & 266 & 0 & 1 & 171 \\
    64 & 1 & 3 & 110 & 211 & 0 & 0 & 144 \\
    \bottomrule[1.5pt]
\end{longtable}
\footnotesize 注:数据来源于Kaggle Heart Attack Analysis \& Prediction Data Set。}

\section{图片}
\label{sec:figure}

\begin{figure}[htb]\centering
    \subfloat[北京大学校徽]{
        \label{sfig:pkulogo-subfloat}
        \includegraphics[height=2cm]{fig/pkulogo.pdf}}\hfil
    \subfloat[北京大学中文校名,依照北京大学标识管理办公室出具的北大标识使用基本规范进行使用]{
        \label{sfig:pkuword-subfloat}
        \includegraphics[height=2cm]{fig/pkuword.pdf}}
    \caption{包含子图形的大图形,使用subfloat}
    \label{fig:example-fig-subfloat}
\end{figure}

\begin{figure}[htb]\centering
    \begin{subfigure}[b]{0.25\linewidth}\centering
        \includegraphics[height=2cm]{fig/pkulogo.pdf}
        \caption{北京大学校徽}
        \label{sfig:pkulogo-subcaption}\end{subfigure}\hfil
    \begin{subfigure}[b]{0.5\linewidth}\centering
        \includegraphics[height=2cm]{fig/pkuword.pdf}
        \caption{北京大学中文校名,依照北京大学标识管理办公室出具的北大标识使用基本规范进行使用}
        \label{sfig:pkuword-subcaption}\end{subfigure}\hfil
    \caption{包含子图形的大图形,使用subcaption}
    \label{fig:example-fig-subcaption}
\end{figure}

当需要插入多个子图的时候,可以选用宏包\verb|subfloat|或\verb|subcaption|,不推荐使用
\verb|subfig|、\verb|subfigure| 和 \verb|subtable|。

若使用宏包\verb|subfloat|或\verb|subcaption|,可直接使用引用\verb|\ref{sfig:xxxx}|引用子图label,如图~\ref{sfig:pkulogo-subfloat}。
否则需要引用主图,再单独标注子图序号,以便符合学位论文要求。

此外,与表格相反,图序和图名需要位于图片的下方。
如果含有子图,每个子图需要具有相应的子图名。

如果需要并排使用两个独立的图形,分别编排图序,则可使用\verb|minipage|,如图~\ref{fig:pkulogo-minipage}和图~\ref{fig:pkuword-minipage}。

\begin{figure}[htb]\centering
    \begin{minipage}{0.25\linewidth}\centering
        \includegraphics[height=2cm]{fig/pkulogo.pdf}
        \caption{北京大学校徽}
        \label{fig:pkulogo-minipage}
    \end{minipage}\hfil
    \begin{minipage}{0.5\linewidth}\centering
        \includegraphics[height=2cm]{fig/pkuword.pdf}
        \caption{北京大学中文校名,依照北京大学标识管理办公室出具的北大标识使用基本规范进行使用}
        \label{fig:pkuword-minipage}
    \end{minipage}
\end{figure}

\section{公式}
\label{sec:equation}

公式部分考虑到写作指南中无关于公式页的说明,并未做改动,使用通用\LaTeX{}规范即可。对于复杂公式需求,可使用\verb|amsmath|宏包结合Mathpix\footnote{\url{https://mathpix.com/}}等自动化识别工具。

\begin{multline*}
\int_a^b\biggl\{\int_a^b[f(x)^2g(y)^2+f(y)^2g(x)^2]
 -2f(x)g(x)f(y)g(y)\,dx\biggr\}\,dy \\
 =\int_a^b\biggl\{g(y)^2\int_a^bf^2+f(y)^2
    \int_a^b g^2-2f(y)g(y)\int_a^b fg\biggr\}\,dy
\end{multline*}

上述公式来源于\citeauthor{liu2003uncertain}的《不确定规划》\citet{liu2003uncertain}。
