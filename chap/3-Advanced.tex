\chapter{高级设置}
本章介绍一些较复杂的设置。

\section{从\CTeX{}宏集继承的功能}
\label{sec:ctex}
pkuthss 文档类建立在 \CTeX{}宏集的 ctexbook\cite{ctex} 文档类之上,
因此,ctexbook 文档类所提供的功能均可以使用。

\subsection{字体设置}

\texttt{pkufontauto} 和 \texttt{pkufontpath} 定义了符合北京大学论文要求的字体,
包括中易宋体、中易黑体、中易楷体和中易仿宋四种字体,具体情况如下:
\begin{itemize}
    \item \verb|\songti|,中易宋体,作为默认中文字体使用,衬线字体,\verb|\textrm|;
    \item \verb|\heiti|,中易黑体,无衬线字体,\verb|\bfseries|,\verb|\textbf|和\verb|\textsf|;
    \item \verb|\fangsong|,中易仿宋,等宽字体,\verb|\texttt|;
    \item \verb|\kaishu|,中易楷体,\verb|\textit|。
\end{itemize}

字形方面,使用中易黑体作为中易宋体的粗体,中易楷体作为中易宋体的斜体,
使用中易黑体、中易楷体、中易仿宋的假粗体作为对应字体的粗体。

\texttt{pkufontauto} 和 \texttt{pkufontpath} 两种选项的区别是:
\texttt{pkufontauto} 从系统中自动搜索字体,适用于 Windows 平台;
\texttt{pkufontpath} 则通过指定字体文件路径,使用当前路径 \texttt{pkufont} 文件夹下的字体,
适用于缺少相应字体的平台。

例如,在 Windows 平台上,应在载入 pkuthss 文档类时加上:
\begin{Verbatim}
\documentclass[fontset=pkufontauto, ...]{pkuthss}
% “...”代表其它的选项。
\end{Verbatim}
而在 Overleaf 平台上,则应在载入 pkuthss 文档类时加上:
\begin{Verbatim}
\documentclass[fontset=pkufontpath, ...]{pkuthss}
% “...”代表其它的选项。
\end{Verbatim}
并在当前路径新建 \texttt{pkufont} 文件夹,放置四种字体。

如果想要更换中文字体,可以通过新建 ctex-fontset-myfontset.def 定义自己的 fontset \verb|myfontset|。
在系统装有相应字体时,也可以使用CTEX预定义的六种中文字库:
\begin{itemize}
    \item \verb|adobe|,使用Adobe公司的中文字体,不支持\hologo{pdfLaTeX}。
    \item \verb|fandol|,使用Fandol 中文字体,不支持\hologo{pdfLaTeX}。
    \item \verb|founder|,使用方正公司的中文字体。
    \item \verb|mac|,使用macOS系统下的字体,不支持\hologo{pdfLaTeX},根据版本又分为macnew和macold两种。
    \begin{itemize}
        \item \verb|macnew|,使用ElCapitan 或之后的多字重华文字体和苹方字体。
        \item \verb|macold|,使用Yosemite 或之前的华文字体。
    \end{itemize}
    \item \verb|ubuntu|,使用Ubuntu系统下的思源宋体、思源黑体和TEX发行版自带的文鼎楷体,不支持\hologo{pdfLaTeX}。
    \item \verb|windows|,使用 Windows 系统下的中易字体和微软雅黑字体。
\end{itemize}
默认情况下,\CTeX{}宏集根据编译方式和操作系统自动指定相应字库。

\subsection{字号设置}

\texttt{zihao} 的选项只有 -4 | 5 | \texttt{false} 三种,
-4 | 5 将文章默认字号 \texttt{\bfseries\string\normalsize}设置为小四号字或五号字,
\texttt{false}禁用本功能。

\subsection{章节新页模式设置}
文档默认情况下是双面模式,每章都从右页(奇数页)开始。
如果希望改成一章可以从任意页开始(禁止章末空白页),可以加上 \texttt{openany} 选项:
\begin{Verbatim}
\documentclass[openany, ...]{pkuthss} % 每章从任意页开始。
\end{Verbatim}

\subsection{论文元素名称设置}
用户可以使用 ctexbook 文档类提供的 \verb|\ctexset| 命令设定论文元素名称:
\begin{Verbatim}
\ctexset{
    appendixname   = {附录},
    bibname        = {参考文献},
    contentsname   = {目录},
    listtablename  = {表格索引},
    listfigurename = {插图索引},
    figurename     = {图},
    tablename      = {表}
}
\end{Verbatim}
例如,将目录的标题改为“目{\quad\quad}录”:
\begin{Verbatim}
\ctexset{
    contentsname = {目\quad\quad录}
}
\end{Verbatim}

\section{从其它宏包继承的功能}
\label{sec:thirdparty}

pkuthss 文档类调用了 geometry\cite{geometry}、fancyhdr\cite{fancyhdr}、%
hyperref\cite{hyperref}、graphicx\cite{graphicx}
和 ulem\cite{ulem} 等几个宏包。
因此,这些宏包所提供的功能均可以使用。

除此之外,pkuthss 文档类还可能调用以下这些宏包:
\begin{itemize}
    \item 启用 \verb|pkufont| 选项时会调用 amsmath、unicode-math
        \cite{unicode-math} 宏包,不启用
        \verb|pkufont| 选项时会调用 amssymb\cite{amssymb} 宏包。
    \item 启用 \verb|pkufoot| 选项时会调用
        tikz\cite{tikz} 和 scrextend\cite{scrextend} 宏包。
    \item 启用 \verb|pkuspace| 选项时会调用
        tocloft\cite{tocloft}、caption\cite{caption} 和
        subcaption\cite{subcaption} 宏包。
    \item 启用 \verb|spacing| 选项时会调用 setspace 和
        enumitem\cite{enumitem} 宏包。
\end{itemize}
因此在启用相应选项时,用户可以使用对应宏包所提供的功能。
