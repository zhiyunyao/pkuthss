\chapter{引言}
\label{chap:intro}

\def\GitHubLink{\href{https://github.com/zhiyunyao/pkuthss/tree/lite}{GitHub仓库链接}}
\def\OverleafLink{\href{https://www.overleaf.com/read/wmsmytgjkxfy\#c888f2}{Overleaf模版链接}}

本文档是北京大学论文模版 \pkuthss{} 的用户指南,基于CasperVector~\cite{casperpkuthss2011}和iofu728~\cite{iofu728pkuthss2021}的文档编写而成。

\pkuthss{}模版参照北京大学研究生院文件《北京大学研究生学位论文写作指南》~\cite{pku-thesisguide}、《博士研究生学位论文格式模板(2024)》~\cite{pku-thesisstyle}和《关于完善研究生学位论文封面的通知》~\cite{pku-thesiscover}编写。
在整合CasperVector/pkuthss~\cite{casperpkuthss2011}和iofu728/pkuthss~\cite{iofu728pkuthss2021}两个模版的基础上,
重写了功能接口,解决了在Overleaf等非Windows平台显示宋体等Windows字体的问题,
同时新增了一些需求比较大的设置功能,如设置封面标题下划线的行数、设置是否在封面显示标题等。

\pkuthss{}模版结构清晰,注释详细,较为易于学习和使用。
希望它能为各位须要使用 \hologo{LaTeX} 撰写论文的同学提供一些帮助。

\GitHubLink \quad \OverleafLink

\section{关键文件}
\begin{itemize}
    \item \verb|thesis.tex|:模版的主文件。
    \item \verb|thesis.bib|:模版的参考文献库。
    \item \verb|pkuthss.cls|:定义pkuthss文档类。
    \item \verb|cover.tex|:生成论文封皮,使用皮纹纸打印,装订在书芯外面。
    \item \verb|ctex-fontset-pkufontauto.def|、\verb|ctex-fontset-pkufontpath.def|:字体配置文件。
    \item \verb|.vscode/settings.json|:VSCode工作区设置文件,设置编译器为\hologo{XeLaTeX},并且不在工作区显示编译中间文件。
    \item \verb|chap/|:各章节内容。
\end{itemize}
\emph{
    注:本模版可排版学校要求的二维码,
    请参考 \texttt{chap/Copyright.tex} 和 \texttt{chap/Declaration.tex} 中的相关注释。
}

\section{编译要求}
\pkuthss{}模版仅支持UTF-8文件编码和\hologo{XeLaTeX}编译。
请确保所有 \verb|tex| 文件为UTF-8编码,并使用\hologo{XeLaTeX}编译。

\section{Quick Start}

\subsection{在VSCode使用}
环境配置:安装TeX Live,配置好LaTeX Workshop扩展。

下载模版:点击\ \GitHubLink\ 访问项目 Zhiyunyao/pkuthss 的 lite 分支,下载zip,解压后用VSCode打开文件夹。注意\verb|.vscode|文件夹应该在打开的工作区根目录下。

使用模版:打开 \verb|thesis.tex|,点击页面右上角空心绿色三角即可编译。

\subsection{在Overleaf使用}
点击\ \OverleafLink\ 访问只读模版项目,Copy该项目到自己的账号下即可使用。

\section{与CasperVector模板不同之处}

\textbf{功能方面:}

\begin{enumerate}[leftmargin=5em]
    \item 盲审模式以文档类选项形式设置
    \item 字体字号以文档类选项形式设置
    \item 重新定义了文档信息的设置接口
    \item 增加表格索引、插图索引
    \item 增加主要符号对照表
    \item 脚注从当前页开始标注
    \item 表格内脚注样式
    \item 子图引用格式
\end{enumerate}

\textbf{格式方面:}

\begin{enumerate}[leftmargin=5em]
    \item ``关键词”、``KEY WORDS” 非粗体 
    \item 默认隐藏超链接
\end{enumerate}
