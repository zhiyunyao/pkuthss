% Zhiyunyao/pkuthss: lite LaTeX template for dissertations at Peking University
% 2024/04/26 v1.9.4-lite
% GitHub:   https://github.com/zhiyunyao/pkuthss/tree/lite
% Overleaf: https://www.overleaf.com/read/wmsmytgjkxfy#c888f2

% Copyright (c) 2008-2009 solvethis
% Copyright (c) 2010-2022,2024 Casper Ti. Vector
% Copyright (c) 2021 Kurapica
% Copyright (c) 2022 iofu728
% Copyright (c) 2024 Zhiyunyao

\documentclass[fontset=pkufontauto,zihao=-4,ugly,openany]{pkuthss}
% 字体库      fontset (pkufontauto | pkufontpath)  系统装有宋体等字体请使用 pkufontauto;
%                                  否则使用 pkufontpath 并新建 pkufont 文件夹放置所需字体
% 默认字号     zihao    (-4 | 5)    设定默认字号为小四|五号
% 学位论文模式  ugly   (默认关闭)   打开后论文严格按照学校格式要求编译
% 盲审模式     blind   (默认关闭)   打开后论文按照盲审格式编译
% 章节新页模式 openany (默认关闭)   每章都从右页(奇数页)开始;打开后每章从任意页开始(禁止章末空白页)

% 参考文献遵循GB/T 7714-2015标准,使用biblatex-gb7714-2015 宏包
% 此处使用顺序编码制,如使用著者-出版年制则更改为b7714-2015ay
\usepackage[backend=biber,style=gb7714-2015]{biblatex}
\renewcommand*{\bibfont}{\zihao{5}\linespread{1.27}\selectfont}
% 设定参考文献列表的段间距
\setlength{\bibitemsep}{3bp}
% 载入参考文献数据库(注意不要省略“.bib”)
\addbibresource{thesis.bib}

% 示例文档用包和设定,该段均可移除
\usepackage[perpage]{footmisc} % 脚注按页编号
\usepackage{enumitem,fancyvrb} % 列表相关
\usepackage{booktabs,multirow,longtable,makecell} % 表格相关
\usepackage{hologo} % Tex徽标
\RecustomVerbatimEnvironment{Verbatim}{Verbatim}{frame=single, tabsize=4, fontsize=\footnotesize}
\renewcommand{\v}[1]{\boldsymbol{#1}}
\newcommand\pkg[1]{\textsf{#1}}
\def\liteversion{v1.9.4-lite}
\def\pkuthss{pkuthss \liteversion}

%% 设定文档基本信息(按照在论文中出现顺序)
% cthesisname: 论文类别,显示在偶数页页眉和pdf文档subject属性,如:
%              北京大学博士学位论文/北京大学硕士学位论文
\cthesisname{北京大学博士学位论文}
% thesiscover: 封面显示的论文类别
\thesiscover{博士研究生学位论文}
% ctitlelines: 封面论文标题的下划线行数,设置为0则不在封面显示标题
\ctitlelines{1}
% ctitle: 论文标题,长标题用“\\”强制换行(v1.9.4-lite之前版本“\\”问题已修复)
\ctitle{\pkuthss{} 用户指南}
\cauthor{某某某}
\studentid{0123456789}
\school{某某学院}
\cmajor{某某专业}
\direction{某某方向}
\cmentorlines{2}
\cmentor{甲\ 教授\\乙\ 教授}
% degreetype: 1->学术学位,2->专业学位,0->不显示学位类型
\degreetype{1}
% date: 具体时间以教务为准,初稿3月,送审4月,答辩5月,最终6月
\date{二〇二四年五月}
\ckeywords{其一,其二}

% 英文信息,只在英文摘要页显示
\etitle{\pkuthss{} User Guide}
\eauthor{XXXXXX XXX}
\emajor{XX Major}
% ementor: 教授 Prof., 副教授 A.P., 讲师 Lec.
\ementor{Prof.\ A and Prof.\ B}
\ekeywords{First, Second}

%% 盲审信息,只在盲审模式显示,无盲审需求的用户可忽略
% discipline: 一级学科(cmajor是二级学科)
\discipline{计算机科学与技术}
% blindid: 盲审论文编号
\blindid{9876543210}

%% 设定论文元素名称
\ctexset{
    % appendixname   = {附录},
    % bibname        = {参考文献},
    % contentsname   = {目录},
    listtablename  = {表格索引},
    listfigurename = {插图索引},
    % figurename     = {图},
    % tablename      = {表}
}

%% 设定链接显示效果
\hypersetup{
    hidelinks,                   % 移除链接的字体颜色和边框
    linktoc            = all,    % 目录设置为链接的级别 (none | section | page | all)
    breaklinks         = true,   % 是否允许链接换行
    pdfdisplaydoctitle = true,   % 是否在文件标题属性展示标题而不是文件名
    bookmarksdepth     = 3,      % pdf 书签最大深度
    bookmarksopen      = true,   % pdf 书签是否自动展开
    bookmarksopenlevel = 1       % pdf 书签自动展开级别
}%

\begin{document}
    %% 以下为正文之前的部分,默认不进行章节编号
    \frontmatter
    % 此后到下一 \pagestyle 命令之前不排版页眉或页脚
    \pagestyle{empty}
    % 自动生成封面
    \maketitle
    % 封面要求单面打印,故须新开右页
    \cleardoublepage
    % 此处不用 \specialchap,因为学校要求目录不包括其自己及其之前的内容。
\chapter*{版权声明}
% 综合学校的书面要求及 Word 模版来看,版权声明页不用加页眉、页脚。
\thispagestyle{empty}

任何收存和保管本论文各种版本的单位和个人,
未经本论文作者同意,不得将本论文转借他人,
亦不得随意复制、抄录、拍照或以任何方式传播。
否则,引起有碍作者著作权之问题,将可能承担法律责任。

% 若须排版二维码,请将二维码图片重命名为“barcode”,
% 转为合适的图片格式,放在 fig 目录下,然后去掉下面 2 行的注释。
% \vfill\noindent
% \includegraphics[height = 5em]{fig/barcode}


    %% 此后到下一 \pagestyle 命令之前正常排版页眉和页脚
    \cleardoublepage
    \pagestyle{plain}
    % 重置页码计数器,用大写罗马数字排版此部分页码
    \setcounter{page}{0}
    \pagenumbering{Roman}
    % 中英文摘要
    \begin{cabstract}
    某某问题是…….
    本文 采用了……
    研究表明…….
\end{cabstract}

\begin{eabstract}
    In environmental economics, environmental resources including environmental quality are categorized as amenity resources. Due to its importance to human welfare, the amenity resources theoretical study and valuation is an ongoing issue at the academic frontier in the environmental economics area.
\end{eabstract}

    % 自动生成目录
    \tableofcontents
    % 如有需要自动生成表格索引、插图索引
    \listoftables
    \listoffigures
    % 如有需要生成主要符号对照表
    \include{chap/Denotation}

    %% 以下为正文部分,默认要进行章节编号
    \mainmatter
    \chapter{引言}
\label{chap:intro}

\def\GitHubLink{\href{https://github.com/zhiyunyao/pkuthss/tree/lite}{GitHub仓库链接}}
\def\OverleafLink{\href{https://www.overleaf.com/read/wmsmytgjkxfy\#c888f2}{Overleaf模版链接}}

本文档是北京大学论文模版 \pkuthss{} 的用户指南,基于CasperVector~\cite{casperpkuthss2011}和iofu728~\cite{iofu728pkuthss2021}的文档编写而成。

\pkuthss{}模版参照北京大学研究生院文件《北京大学研究生学位论文写作指南》~\cite{pku-thesisguide}、《博士研究生学位论文格式模板(2024)》~\cite{pku-thesisstyle}和《关于完善研究生学位论文封面的通知》~\cite{pku-thesiscover}编写。
在整合CasperVector/pkuthss~\cite{casperpkuthss2011}和iofu728/pkuthss~\cite{iofu728pkuthss2021}两个模版的基础上,
重写了功能接口,解决了在Overleaf等非Windows平台显示宋体等Windows字体的问题,
同时新增了一些需求比较大的设置功能,如设置封面标题下划线的行数、设置是否在封面显示标题等。

\pkuthss{}模版结构清晰,注释详细,较为易于学习和使用。
希望它能为各位须要使用 \hologo{LaTeX} 撰写论文的同学提供一些帮助。

\GitHubLink \quad \OverleafLink

\section{关键文件}
\begin{itemize}
    \item \verb|thesis.tex|:模版的主文件。
    \item \verb|thesis.bib|:模版的参考文献库。
    \item \verb|pkuthss.cls|:定义pkuthss文档类。
    \item \verb|cover.tex|:生成论文封皮,使用皮纹纸打印,装订在书芯外面。
    \item \verb|ctex-fontset-pkufontauto.def|、\verb|ctex-fontset-pkufontpath.def|:字体配置文件。
    \item \verb|.vscode/settings.json|:VSCode工作区设置文件,设置编译器为\hologo{XeLaTeX},并且不在工作区显示编译中间文件。
    \item \verb|chap/|:各章节内容。
\end{itemize}
\emph{
    注:本模版可排版学校要求的二维码,
    请参考 \texttt{chap/Copyright.tex} 和 \texttt{chap/Declaration.tex} 中的相关注释。
}

\section{编译要求}
\pkuthss{}模版仅支持UTF-8文件编码和\hologo{XeLaTeX}编译。
请确保所有 \verb|tex| 文件为UTF-8编码,并使用\hologo{XeLaTeX}编译。

\section{Quick Start}

\subsection{在VSCode使用}
环境配置:安装TeX Live,配置好LaTeX Workshop扩展。

下载模版:点击\ \GitHubLink\ 访问项目 Zhiyunyao/pkuthss 的 lite 分支,下载zip,解压后用VSCode打开文件夹。注意\verb|.vscode|文件夹应该在打开的工作区根目录下。

使用模版:打开 \verb|thesis.tex|,点击页面右上角空心绿色三角即可编译。

\subsection{在Overleaf使用}
点击\ \OverleafLink\ 访问只读模版项目,Copy该项目到自己的账号下即可使用。

\section{与CasperVector模板不同之处}

\textbf{功能方面:}

\begin{enumerate}[leftmargin=5em]
    \item 盲审模式以文档类选项形式设置
    \item 字体字号以文档类选项形式设置
    \item 重新定义了文档信息的设置接口
    \item 增加表格索引、插图索引
    \item 增加主要符号对照表
    \item 脚注从当前页开始标注
    \item 表格内脚注样式
    \item 子图引用格式
\end{enumerate}

\textbf{格式方面:}

\begin{enumerate}[leftmargin=5em]
    \item ``关键词”、``KEY WORDS” 非粗体 
    \item 默认隐藏超链接
\end{enumerate}

    \chapter{模版功能}
本章对模版提供的功能和配置项进行介绍说明。

\section{文档类选项}
\label{sec:option}

除非特别说明,否则这一节提到的选项中都是不带“\verb|no|”的版本被启用。

\begin{itemize}
    \item \textbf{\texttt{[no]blind}}:
        是否按照盲审格式编译,使用盲审封面,不包含“发表论文信息”、“致谢”和“声明”章节。\emph{%
            pkuthss 文档类默认启用 \texttt{noblind} 选项,
            即不按照盲审格式编译%
        }。

    \item \textbf{\texttt{[no]uppermark}}:
        是否在页眉中将章节名中的小写字母转换为大写字母。
        就目前而言,这样的转换存在着一些较为严重的缺陷\footnote{%
            准确地说是 \texttt{\string\MakeUppercase} 宏的问题:
            其在某些地方的转换不够健壮,
            例如 \texttt{\string\cite\string{ctex\string}}
            会被转换成 \texttt{\string\cite\string{CTEX\string}}。%
        },因此不建议使用。
        基于上述考虑,\emph{%
            pkuthss 文档类默认启用 \texttt{nouppermark} 选项,
            即不在页眉中使用大写的章节名%
        }。

    \item \textbf{\texttt{[no]pkufont}}:
        是否根据学校对论文格式的要求\cite{pku-thesisstyle}%
        将西文字体改为类似于 Times New Roman / Arial 的字体。

    \item \textbf{\texttt{[no]pkufoot}}\footnote{%
            此选项等价于 1.6.4 及以前版本 pkuthss-extra 宏包的
            \texttt{[no]footfmt} 选项;
            更改名称是为了使文档类选项名更加规则。%
        }:
        是否根据学校对论文格式的要求\cite{pku-thesisstyle}%
        修改和脚注相关的一些格式。
        具体地说,启用 \verb|pkufoot| 选项后会进行以下几项设置:
    \begin{itemize}
        \item 脚注参用带圈的编号。
        \item 页脚中脚注编号使用正文(而非上标)字体。
        \item 页脚中脚注编号和脚注文本之间默认间隔一个空格。
    \end{itemize}

    \item \textbf{\texttt{[no]pkuspace}}:
        是否根据学校对论文格式的要求\cite{pku-thesisstyle}%
        修改排版中的一些间距及相关设置。
        具体地说,启用 \verb|pkuspace| 选项后会按
        \parencite{pku-thesisstyle} 中的要求修改以下几项设置:
    \begin{itemize}
        \item 正文的行距。
        \item 目录中条目的缩进方式。
        \item 图表标题的字号,以及标题中编号和标题文字之间的间隔方式
    \end{itemize}

    \item \textbf{\texttt{[no]spacing}}\footnote{%
            因为代码重构的缘故,
            此选项同时提供 1.5.5 及以前版本 pkuthss-extra 宏包
            中 \texttt{[no]tightlist} 选项所提供的功能。%
        }:
        是否采用一些常用的调整间距的额外版式设定。
        具体地说,启用 \verb|spacing| 选项后会进行以下几项设置:
    \begin{itemize}
        \item 调用 setspace 宏包以使某些细节处的空间安排更美观。
        \item 采用比 \hologo{LaTeX} 默认设定更加紧密的枚举环境%
            \footnote{%
                在枚举环境(itemize、enumerate 和 description)中,
                每个条目的内容较少时,条目往往显得稀疏;
                在参考文献列表中也有类似的现象。
                启用 \texttt{spacing} 选项后,
                将去掉这些环境中额外增加的(垂直)间隔。%
            }。
        \item 调整枚举环境的缩进,以适应中文排版中的习惯。
    \end{itemize}

    \item \textbf{\texttt{[no]spechap}}\footnote{%
            “spechap”是“\textbf{spec}ial \textbf{chap}ter”的缩写。%
        }:
        是否启用第 \ref{sec:misc} 小节中介绍的 \verb|\specialchap| 命令。

    \item \textbf{\texttt{[no]pdftoc}}\footnote{%
            此选项部分等价于 1.4 alpha2 及以前版本 pkuthss-extra 宏包
            的 \texttt{[no]tocbibind} 选项。
            因为 tocbibind 宏包和 biblatex 宏包冲突,%
            pkuthss 文档类不再调用 tocbibind 宏包。%
        }:
        启用 \verb|pdftoc| 选项后,
        用 \verb|\tableofcontents| 命令生成目录时会自动添加“目录”的 pdf 书签。

    \item \textbf{\texttt{[no]pdfprop}}:
        是否自动根据设定的论文文档信息(如作者、标题等)
        设置生成的 pdf 文档的相应属性。\emph{%
            注意:该选项实际上是在 \texttt{\string\maketitle} 时生效的,
            这是因为考虑到
            通常用户在调用 \texttt{\string\maketitle} 前
            已经设置好所有的文档信息。
            若用户不调用 \texttt{\string\maketitle},
            则须在设定完文档信息之后自行调用
            第 \ref{sec:misc} 小节中介绍的
            \texttt{\string\setpdfproperties} 命令以完成
            pdf 文档属性的设定。%
        }

    \item \textbf{其余文档类选项}:%
        pkuthss 文档类以 ctexbook 文档类为基础,
        其接受的其余所有文档类选项均被传递给 ctexbook。
        其中可能最常用的选项是 \verb|fontset| 和 \verb|zihao|,
        它们选择中文字体和默认字号。详见第 \ref{sec:ctex} 小节。
\end{itemize}

\section{文档信息设定}
\label{sec:info}

这一类命令的语法为
\begin{Verbatim}
  \commandname{具体信息} % commandname 为具体命令的名称。
\end{Verbatim}

这些命令总结如下:
\begin{itemize}
    \item \texttt{\bfseries\string\cthesisname}:论文类别的中文名;
    \item \texttt{\bfseries\string\thesiscover}:封面显示的论文类别;
    \item \texttt{\bfseries\string\ctitlelines}:封面论文标题的下划线行数,设置为0则不在封面显示标题;
    \item \texttt{\bfseries\string\ctitle}:设定论文中文标题,长标题用“\backslash\backslash”强制换行;
    \item \texttt{\bfseries\string\cauthor}:设定作者的中文名;
    \item \texttt{\bfseries\string\studentid}:设定作者的学号;
    \item \texttt{\bfseries\string\school}:设定作者的学院名;
    \item \texttt{\bfseries\string\cmajor}:设定作者专业(二级学科)的中文名;
    \item \texttt{\bfseries\string\direction}:设定作者的研究方向;
    \item \texttt{\bfseries\string\cmentorlines}:封面“导师”部分的行数;
    \item \texttt{\bfseries\string\cmentor}:设定导师的中文名;
    \item \texttt{\bfseries\string\degreetype}:
        设定学位类型(1 为学术学位,2 为专业学位,设为 0 则不显示学位类型);
    \item \texttt{\bfseries\string\date}:设定日期;
    \item \texttt{\bfseries\string\ckeywords}:设定中文关键词;
    \item \texttt{\bfseries\string\etitle}:设定论文西文标题;
    \item \texttt{\bfseries\string\eauthor}:设定作者的西文名;
    \item \texttt{\bfseries\string\emajor}:设定作者专业(二级学科)的西文名;
    \item \texttt{\bfseries\string\ementor}:设定导师的西文名;
    \item \texttt{\bfseries\string\ekeywords}:设定西文关键词;
    \item \texttt{\bfseries\string\discipline}:设定一级学科(双盲评审用)。
    \item \texttt{\bfseries\string\blindid}:设定论文编号(双盲评审用);
\end{itemize}

例如,如果要设定专业为“化学”(“Chemistry”),则可以使用以下命令:
\begin{Verbatim}
\cmajor{化学}
\emajor{Chemistry}
\end{Verbatim}

\section{摘要}
\label{sec:abstract}

\texttt{\bfseries cabstract} 和 \texttt{\bfseries eabstract} 环境用于编写
中英文摘要。
用户只须要写摘要的正文;标题、作者、导师、专业等部分会自动生成,盲审模式下这些信息也会自动隐藏。

如论文工作受到基金资助,需要在中文摘要第一页的页脚处标注:本研 究得到某某基金(编号:xxx)资助。


\section{目录、表格索引、插图索引}
\label{sec:directory}

目录使用 \texttt{\bfseries \string\tableofcontents} 命令生成。
表格索引使用 \texttt{\bfseries \string\listoftables} 命令生成。
插图索引使用 \texttt{\bfseries \string\listoffigures} 命令生成。

\section{主要符号对照表}
\label{sec:denotation}

参考\verb|chap/Denotation.tex|即可,在\verb|denotation|环境下,使用\verb|\item[X] Y|分别表示符号及其说明。

已知问题: 符号处不能输入中括号$[$,$]$。

\section{参考文献}
\label{sec:bibtex}

参考文献根据写作指南使用\verb|gb7714-2015|bibstyle进行管理,具体引用命令与日常使用类似,\verb|\cite{}|,\verb|\citet{}|,\verb|\citeauthor{}|,具体用法见相应文档\footnote{\url{https://github.com/hushidong/biblatex-gb7714-2015}}。

例如\verb|\cite{devlin2018bert}|=\cite{devlin2018bert},\verb|\citeauthor{gut2013probability}|=\citeauthor{gut2013probability},...
相对于的bib文件的书写基本上直接用Google Scholar拷贝的BibTex即可,部分属性按提示进行微调。
\begin{Verbatim}
    \usepackage[backend=biber,style=gb7714-2015]{biblatex}
\end{Verbatim}

\section{其他}
\label{sec:misc}

《北京大学研究生学位论文写作指南》~\cite{pku-thesisguide}提到“一般情况下,不建议使用三级及以上节标题”。
对应到本模板,即不建议使用命令 \texttt{\bfseries\string\subsubsection\{\}}。

\texttt{\bfseries\string\specialchap\{\}} 命令
用于开始不进行标号但计入目录的一章,
并合理安排其页眉。\emph{%
    注意:须要启用 \texttt{spechap} 选项才能使用此命令。
    另外,在此章内的节或小节等命令应使用带星号的版本,
    例如 \texttt{\string\section\string*\{\}} 等,
    以免造成章节编号混乱。%
}%
例如,本文档中的“北京大学学位论文原创性声明和使用授权说明”一章就是用 \verb|\specialchap{北京大学学位论文原创性声明和使用授权说明}|
这条命令开始的。%

\texttt{\bfseries\string\setpdfproperties} 命令
用于根据用户设定的文档信息自动设定生成的 pdf 文档的属性。
此命令会在用户调用 \verb|\maketitle| 命令时被自动调用,
因此通常不需要用户自己使用;
但用户有时可能不须要输出封面,从而不会调用 \verb|\maketitle| 命令,
此时就须要在设定完文档信息之后调用 \verb|\setpdfproperties|。%
\emph{注意:须要启用 \texttt{pdfprop} 选项才能使用此命令。}

    \chapter{高级设置}
本章介绍一些较复杂的设置。

\section{从\CTeX{}宏集继承的功能}
\label{sec:ctex}
pkuthss 文档类建立在 \CTeX{}宏集的 ctexbook\cite{ctex} 文档类之上,
因此,ctexbook 文档类所提供的功能均可以使用。

\subsection{字体设置}

\texttt{pkufontauto} 和 \texttt{pkufontpath} 定义了符合北京大学论文要求的字体,
包括中易宋体、中易黑体、中易楷体和中易仿宋四种字体,具体情况如下:
\begin{itemize}
    \item \verb|\songti|,中易宋体,作为默认中文字体使用,衬线字体,\verb|\textrm|;
    \item \verb|\heiti|,中易黑体,无衬线字体,\verb|\bfseries|,\verb|\textbf|和\verb|\textsf|;
    \item \verb|\fangsong|,中易仿宋,等宽字体,\verb|\texttt|;
    \item \verb|\kaishu|,中易楷体,\verb|\textit|。
\end{itemize}

字形方面,使用中易黑体作为中易宋体的粗体,中易楷体作为中易宋体的斜体,
使用中易黑体、中易楷体、中易仿宋的假粗体作为对应字体的粗体。

\texttt{pkufontauto} 和 \texttt{pkufontpath} 两种选项的区别是:
\texttt{pkufontauto} 从系统中自动搜索字体,适用于 Windows 平台;
\texttt{pkufontpath} 则通过指定字体文件路径,使用当前路径 \texttt{pkufont} 文件夹下的字体,
适用于缺少相应字体的平台。

例如,在 Windows 平台上,应在载入 pkuthss 文档类时加上:
\begin{Verbatim}
\documentclass[fontset=pkufontauto, ...]{pkuthss}
% “...”代表其它的选项。
\end{Verbatim}
而在 Overleaf 平台上,则应在载入 pkuthss 文档类时加上:
\begin{Verbatim}
\documentclass[fontset=pkufontpath, ...]{pkuthss}
% “...”代表其它的选项。
\end{Verbatim}
并在当前路径新建 \texttt{pkufont} 文件夹,放置四种字体。

如果想要更换中文字体,可以通过新建 ctex-fontset-myfontset.def 定义自己的 fontset \verb|myfontset|。
在系统装有相应字体时,也可以使用CTEX预定义的六种中文字库:
\begin{itemize}
    \item \verb|adobe|,使用Adobe公司的中文字体,不支持\hologo{pdfLaTeX}。
    \item \verb|fandol|,使用Fandol 中文字体,不支持\hologo{pdfLaTeX}。
    \item \verb|founder|,使用方正公司的中文字体。
    \item \verb|mac|,使用macOS系统下的字体,不支持\hologo{pdfLaTeX},根据版本又分为macnew和macold两种。
    \begin{itemize}
        \item \verb|macnew|,使用ElCapitan 或之后的多字重华文字体和苹方字体。
        \item \verb|macold|,使用Yosemite 或之前的华文字体。
    \end{itemize}
    \item \verb|ubuntu|,使用Ubuntu系统下的思源宋体、思源黑体和TEX发行版自带的文鼎楷体,不支持\hologo{pdfLaTeX}。
    \item \verb|windows|,使用 Windows 系统下的中易字体和微软雅黑字体。
\end{itemize}
默认情况下,\CTeX{}宏集根据编译方式和操作系统自动指定相应字库。

\subsection{字号设置}

\texttt{zihao} 的选项只有 -4 | 5 | \texttt{false} 三种,
-4 | 5 将文章默认字号 \texttt{\bfseries\string\normalsize}设置为小四号字或五号字,
\texttt{false}禁用本功能。

\subsection{章节新页模式设置}
文档默认情况下是双面模式,每章都从右页(奇数页)开始。
如果希望改成一章可以从任意页开始(禁止章末空白页),可以加上 \texttt{openany} 选项:
\begin{Verbatim}
\documentclass[openany, ...]{pkuthss} % 每章从任意页开始。
\end{Verbatim}

\subsection{论文元素名称设置}
用户可以使用 ctexbook 文档类提供的 \verb|\ctexset| 命令设定论文元素名称:
\begin{Verbatim}
\ctexset{
    appendixname   = {附录},
    bibname        = {参考文献},
    contentsname   = {目录},
    listtablename  = {表格索引},
    listfigurename = {插图索引},
    figurename     = {图},
    tablename      = {表}
}
\end{Verbatim}
例如,将目录的标题改为“目{\quad\quad}录”:
\begin{Verbatim}
\ctexset{
    contentsname = {目\quad\quad录}
}
\end{Verbatim}

\section{从其它宏包继承的功能}
\label{sec:thirdparty}

pkuthss 文档类调用了 geometry\cite{geometry}、fancyhdr\cite{fancyhdr}、%
hyperref\cite{hyperref}、graphicx\cite{graphicx}
和 ulem\cite{ulem} 等几个宏包。
因此,这些宏包所提供的功能均可以使用。

除此之外,pkuthss 文档类还可能调用以下这些宏包:
\begin{itemize}
    \item 启用 \verb|pkufont| 选项时会调用 amsmath、unicode-math
        \cite{unicode-math} 宏包,不启用
        \verb|pkufont| 选项时会调用 amssymb\cite{amssymb} 宏包。
    \item 启用 \verb|pkufoot| 选项时会调用
        tikz\cite{tikz} 和 scrextend\cite{scrextend} 宏包。
    \item 启用 \verb|pkuspace| 选项时会调用
        tocloft\cite{tocloft}、caption\cite{caption} 和
        subcaption\cite{subcaption} 宏包。
    \item 启用 \verb|spacing| 选项时会调用 setspace 和
        enumitem\cite{enumitem} 宏包。
\end{itemize}
因此在启用相应选项时,用户可以使用对应宏包所提供的功能。

    \chapter{常见需求实现}
本章介绍一些模板功能之外的常见需求的实现方法。
 
\section{表格}
\label{sec:table}

一般学术论文需要使用三线表(如表~\ref{tab:example-table-basic}),需要依赖宏包\verb|booktabs|,使用\verb|\toprule|,\verb|\midrule|,\verb|\bottomrule|控制三线。
此外表序和表名位于表格的上方。
如果需要对表格内进行脚注,可通过\texttt{minipage}中嵌套\texttt{tabular}来实现,具体可参考Stack Overflow\footnote{\url{https://stackoverflow.com/questions/2888817/footnotes-for-tables-in-latex}}。

如需要注明表格中数据来源,则可使用类似的方式,见表~\ref{tab:example-table-source-foot}。

\begin{table*}[htb]
    \centering
    \begin{minipage}[t]{0.55\linewidth} %
        \caption[表格脚注样例表]{表格脚注样例表。表名可通过中括号添加缩略名。}
        \label{tab:example-table-basic}
        \begin{small}
        \begin{tabular}{@{}lccccc@{}}
         \toprule[1.5pt]
         & \textbf{X} & \textbf{Y} & \textbf{Z} & \textbf{N} & \textbf{M} \\
         \midrule[1pt]
            默认        & 99.99 & 99.99 & 99.99 & 99.99\footnote{表格中的脚注1} & 99.99 \\
          \quad w/o X   & 99.99 & 99.99 & 99.99 & 99.99 & 99.99 \\
          \quad w/o Y   & 99.99 & 99.99 & 99.99 & 99.99 & 99.99 \\
          \quad w/o Z   & 99.99\footnote{表格中的脚注2} & 99.99 & 99.99 & 99.99 & 99.99 \\
          \quad w/o N   & 99.99 & 99.99 & 99.99 & 99.99 & 99.99 \\
          \quad w/o M   & 99.99 & 99.99 & 99.99 & 99.99 & 99.99 \\
          \bottomrule[1.5pt]
        \end{tabular}
        \end{small}
    \end{minipage}
\end{table*}

\begin{table*}[htbp]
   \centering
   \caption[数据来源注释表]{表格数据来源注释样例表。}
   \label{tab:example-table-source-foot}
   \begin{minipage}[t]{0.9\textwidth}
   \begin{small}
   \begin{tabular}{@{}l|ccc|ccc@{}}
   \toprule
   \multirow{2}{*}{\textbf{Model}} & \multicolumn{3}{c|}{\textbf{数据集A}} & \multicolumn{3}{c}{\textbf{数据集B}} \\ \cmidrule(l){2-7} 
    & \textbf{指标a}(\%) & \textbf{指标b}(\%) & \textbf{指标c} & \textbf{指标a} (\%) & \textbf{指标b}(\%) & \textbf{指标c} \\ \midrule
      \citet{devlin2018bert}      &99.99  & 99.99  & 99.99  &99.99  & 99.99  & 99.99  \\
      \citet{yang2019xlnet}      &99.99  & 99.99  & 99.99  &99.99  & 99.99  & 99.99  \\
    \bottomrule
   \end{tabular}\\[6pt]
   \footnotesize 注:数据来源XXXXXX。\\
   \end{small}
   \end{minipage}
\end{table*}

当表格较大,不能在一页内打印时,可以“续表”的形式另页打印,可使用宏包\verb|longtable|实现,如表~\ref{tab:example-table-continue}。

{\begin{longtable}[c]{c*{7}{r}}
    \caption[续表]{续表样例表。}
    \label{tab:example-table-continue}\\
    \toprule[1.5pt]
     \multicolumn{1}{c}{年龄} & 性别 & \multicolumn{1}{c}{cp} & \multicolumn{1}{c}{静息血压} & \multicolumn{1}{c}{chol}
    & \multicolumn{1}{c}{空腹血糖>} & \multicolumn{1}{c}{restecg} & \multicolumn{1}{c}{thalachh} \\
    \multicolumn{1}{c}{(岁)} & & \multicolumn{1}{c}{胸痛型}&
    \multicolumn{1}{c}{毫米汞柱}& \multicolumn{1}{c}{胆固醇}& \multicolumn{1}{c}{
       120 mg/dl}& 静息状态 & 最大心率 \\\midrule[1pt]
    \endfirsthead
    \multicolumn{8}{c}{续表~\thetable\hskip1em 续表样例表。}\\
    \toprule[1.5pt]
     \multicolumn{1}{c}{年龄} & 性别 & \multicolumn{1}{c}{cp} & \multicolumn{1}{c}{静息血压} & \multicolumn{1}{c}{chol}
    & \multicolumn{1}{c}{空腹血糖>} & \multicolumn{1}{c}{restecg} & \multicolumn{1}{c}{thalachh} \\
    \multicolumn{1}{c}{(岁)} & & \multicolumn{1}{c}{胸痛型}&
    \multicolumn{1}{c}{毫米汞柱}& \multicolumn{1}{c}{胆固醇}& \multicolumn{1}{c}{
       120 mg/dl}& 静息状态 & 最大心率 \\\midrule[1pt]
    \endhead
    \hline
    \multicolumn{8}{r}{续下页}
    \endfoot
    \endlastfoot
    63 & 1 & 3 & 145 & 233 & 1 & 0 & 150 \\
    37 & 1 & 2 & 130 & 250 & 0 & 1 & 187 \\
    41 & 0 & 1 & 130 & 204 & 0 & 0 & 172 \\
    56 & 1 & 1 & 120 & 236 & 0 & 1 & 178 \\
    57 & 0 & 0 & 120 & 354 & 0 & 1 & 163 \\
    57 & 1 & 0 & 140 & 192 & 0 & 1 & 148 \\
    56 & 0 & 1 & 140 & 294 & 0 & 0 & 153 \\
    44 & 1 & 1 & 120 & 263 & 0 & 1 & 173 \\
    52 & 1 & 2 & 172 & 199 & 1 & 1 & 162 \\
    57 & 1 & 2 & 150 & 168 & 0 & 1 & 174 \\
    54 & 1 & 0 & 140 & 239 & 0 & 1 & 160 \\
    48 & 0 & 2 & 130 & 275 & 0 & 1 & 139 \\
    49 & 1 & 1 & 130 & 266 & 0 & 1 & 171 \\
    64 & 1 & 3 & 110 & 211 & 0 & 0 & 144 \\
    63 & 1 & 3 & 145 & 233 & 1 & 0 & 150 \\
    37 & 1 & 2 & 130 & 250 & 0 & 1 & 187 \\
    41 & 0 & 1 & 130 & 204 & 0 & 0 & 172 \\
    56 & 1 & 1 & 120 & 236 & 0 & 1 & 178 \\
    57 & 0 & 0 & 120 & 354 & 0 & 1 & 163 \\
    57 & 1 & 0 & 140 & 192 & 0 & 1 & 148 \\
    56 & 0 & 1 & 140 & 294 & 0 & 0 & 153 \\
    44 & 1 & 1 & 120 & 263 & 0 & 1 & 173 \\
    52 & 1 & 2 & 172 & 199 & 1 & 1 & 162 \\
    57 & 1 & 2 & 150 & 168 & 0 & 1 & 174 \\
    54 & 1 & 0 & 140 & 239 & 0 & 1 & 160 \\
    48 & 0 & 2 & 130 & 275 & 0 & 1 & 139 \\
    49 & 1 & 1 & 130 & 266 & 0 & 1 & 171 \\
    64 & 1 & 3 & 110 & 211 & 0 & 0 & 144 \\
    49 & 1 & 1 & 130 & 266 & 0 & 1 & 171 \\
    64 & 1 & 3 & 110 & 211 & 0 & 0 & 144 \\
    \bottomrule[1.5pt]
\end{longtable}
\footnotesize 注:数据来源于Kaggle Heart Attack Analysis \& Prediction Data Set。}

\section{图片}
\label{sec:figure}

\begin{figure}[htb]\centering
    \subfloat[北京大学校徽]{
        \label{sfig:pkulogo-subfloat}
        \includegraphics[height=2cm]{fig/pkulogo.pdf}}\hfil
    \subfloat[北京大学中文校名,依照北京大学标识管理办公室出具的北大标识使用基本规范进行使用]{
        \label{sfig:pkuword-subfloat}
        \includegraphics[height=2cm]{fig/pkuword.pdf}}
    \caption{包含子图形的大图形,使用subfloat}
    \label{fig:example-fig-subfloat}
\end{figure}

\begin{figure}[htb]\centering
    \begin{subfigure}[b]{0.25\linewidth}\centering
        \includegraphics[height=2cm]{fig/pkulogo.pdf}
        \caption{北京大学校徽}
        \label{sfig:pkulogo-subcaption}\end{subfigure}\hfil
    \begin{subfigure}[b]{0.5\linewidth}\centering
        \includegraphics[height=2cm]{fig/pkuword.pdf}
        \caption{北京大学中文校名,依照北京大学标识管理办公室出具的北大标识使用基本规范进行使用}
        \label{sfig:pkuword-subcaption}\end{subfigure}\hfil
    \caption{包含子图形的大图形,使用subcaption}
    \label{fig:example-fig-subcaption}
\end{figure}

当需要插入多个子图的时候,可以选用宏包\verb|subfloat|或\verb|subcaption|,不推荐使用
\verb|subfig|、\verb|subfigure| 和 \verb|subtable|。

若使用宏包\verb|subfloat|或\verb|subcaption|,可直接使用引用\verb|\ref{sfig:xxxx}|引用子图label,如图~\ref{sfig:pkulogo-subfloat}。
否则需要引用主图,再单独标注子图序号,以便符合学位论文要求。

此外,与表格相反,图序和图名需要位于图片的下方。
如果含有子图,每个子图需要具有相应的子图名。

如果需要并排使用两个独立的图形,分别编排图序,则可使用\verb|minipage|,如图~\ref{fig:pkulogo-minipage}和图~\ref{fig:pkuword-minipage}。

\begin{figure}[htb]\centering
    \begin{minipage}{0.25\linewidth}\centering
        \includegraphics[height=2cm]{fig/pkulogo.pdf}
        \caption{北京大学校徽}
        \label{fig:pkulogo-minipage}
    \end{minipage}\hfil
    \begin{minipage}{0.5\linewidth}\centering
        \includegraphics[height=2cm]{fig/pkuword.pdf}
        \caption{北京大学中文校名,依照北京大学标识管理办公室出具的北大标识使用基本规范进行使用}
        \label{fig:pkuword-minipage}
    \end{minipage}
\end{figure}

\section{公式}
\label{sec:equation}

公式部分考虑到写作指南中无关于公式页的说明,并未做改动,使用通用\LaTeX{}规范即可。对于复杂公式需求,可使用\verb|amsmath|宏包结合Mathpix\footnote{\url{https://mathpix.com/}}等自动化识别工具。

\begin{multline*}
\int_a^b\biggl\{\int_a^b[f(x)^2g(y)^2+f(y)^2g(x)^2]
 -2f(x)g(x)f(y)g(y)\,dx\biggr\}\,dy \\
 =\int_a^b\biggl\{g(y)^2\int_a^bf^2+f(y)^2
    \int_a^b g^2-2f(y)g(y)\int_a^b fg\biggr\}\,dy
\end{multline*}

上述公式来源于\citeauthor{liu2003uncertain}的《不确定规划》\citet{liu2003uncertain}。


    %% 正文中的附录部分
    \appendix
    % 要使参考文献列表参与章节编号,可将“bibintoc”改为“bibnumbered”
    \printbibliography[heading=bibintoc]
    \chapter{附录示例}


    %% 以下为正文之后的部分,默认不进行章节编号
    \backmatter
    \ifnoblind
        \include{chap/Publication}
        \chapter{致谢}
本论文是在xx老师的悉心指导下完成的。
xx老师作为一名优秀的、经验丰富的教师,具有丰富的xx知识和xx经验,
在整个论文实验和论文写作过程中,对我进行了耐心的指导和帮助,提出严格要求,
引导我不断开阔思路,为我答疑解惑,鼓励我大胆创新,使我在这一段宝贵的时光中,
既增长了知识、开阔了视野、锻炼了心态,又培养了良好的实验习惯和科研精神。
在此,我向我的指导老师表示最诚挚的谢意!

……

(仅为网络示例,可根据论文实际进行撰写,使用时把模版示例内容尽皆删除即可)

        \include{chap/Declaration}
    \fi

\end{document}
